\chapter{Further development}

%The refactoring does not support \code{try} statements in the method body, given the complex control flow involving
%these statements. They can however be supported by the refactoring with some more effort, but they are not usually used
%in the context of recursive methods. Recursive calls inside \code{switch} statements are not supported either, because
%they are rather uncommon. The \code{switch} statement could be however supported by first transforming it into a chain
%of equivalent \code{if} statements before applying the refactoring.

The refactoring does not support recursive calls inside \code{try} statements. This is a rather obscure use case,
because the exception the \code{try} statement would catch should be generated somewhere else inside the same method.
No real usable code would be written that way.

There is basic and unstable support for recursive calls inside \code{switch} statements. They are supported by
replacing each block in the body of the original \code{switch} statement with jumps to the basic blocks corresponding
to these blocks. The original \code{switch} statement thus becomes a new terminator statement for the basic block inside
which the statement appears.

A special case to be considered is \texit{tail recursion}. In the case of tail recursive methods, which are methods
whose last statement is a recursive call, the recursion can be removed by placing the body of the method inside a
\code{while} loop and then replacing the \code{return} statements with assignments to the formal parameters of the
method, ensuring that the order of assignments respects the order of evaluation of the arguments to the recursive call.
Auxiliary variables can be used here, if necessary.

In the case of tail recursive methods with \code{void} return type, the transformation is more complicated because it
is difficult to determine the last statement executed on each code path in the method body, as opposed to methods with
no \code{void} return type, where the last statements executed on each code path are explicit \code{return} statements.