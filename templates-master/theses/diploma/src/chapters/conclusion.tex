\chapter{Conclusion}

%The algorithm for transforming recursion in iteration is composed of thirteen passes which gradually transform the body
%of the method. The end result is a method which is functionally equivalent to the initial method, but the recursion does
%not take place implicitly anymore, but explicitly by simulating it in the user program with a stack of frames. This is
%how \code{StackOverflowError}s can be avoided, at the cost of obfuscating the code.
%
%The performance of the generated code is generally worse than the performance of the original code, but this is
%generally acceptable, because this refactoring should only be used as a temporary solution to
%\code{StackOverflowError}s.

The purpose of this refactoring is to offer a temporary solution for \code{StackOverflowError}s. This is achieved by
simulating the call stack in the user program and building a reduced version of the control flow graph for the recursive
method, which enables jumps to arbitrary places in the code. The process of embedding the control flow graph inside
a \code{switch} and a \code{while} statement is necessary, given the limitation of the Java programming language of not
supporting the \code{goto} statement, which would have enabled jumps to any point in the method body. The refactoring
does not improve either the performance or the readability of the original code of the recursive method. The algorithm
still needs to be improved if performance is critical.

The most difficult part of the algorithm is preserving the semantics of the original code. Given the fact that there is
no formal proof for the corectness of the transformations and that not all the Java language features are supported,
there is no strong guarantee about the semantic equivalence. Writing a refactoring that is accurate in almost any
situation is difficult, because there are many use cases which are hard to predict beforehand.

The refactoring has been tested on about twenty different recursive methods in order to confirm its correctness in the
most common use cases, by checking their behaviour before and after the refactoring. The user of the refactoring should
have a way of testing the recursive method before and after applying the refactoring, probably by writing some unit
tests, in order to be sure the semantics are preserved.