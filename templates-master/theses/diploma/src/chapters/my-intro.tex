\chapter{Introduction}

The purpose of this thesis is to present an algorithm for transforming a recursive method in the Java programming
language into a method which simulates the recursion using an explicit stack in the user program. This transformation
preserves the semantics of the original code, but since the main purpose of this refactoring is to avoid
\code{StackOverflowError}s, the performance of the transformed method may be worse than the performance of the original
one. This refactoring is a temporary solution for the bigger problem of improving the original algorithm.

The algorithm of the refactoring is split into multiple steps. The main idea of the algorithm is to construct a reduced
version of the control flow graph of the method, which highlights the points of return from recursion in the method
body. These points need to be highlighted because control flow has to jump there after returning from a recursive call
and continue executing the following statements. Because the Java programming language does not support the \code{goto}
statement, jumping into the method body is accomplished by enclosing the basic blocks of the control flow graph inside
a \code{switch} statement, each block having its corresponding \code{case} label. The \code{switch} statement itself
has to be placed inside a \code{while} statement, which loops until the simulated stack gets empty.

The reduced control flow graph contains basic blocks corresponding only to statements which contain at least one
recursive call. The statements which do not contain recursive calls do not get transformed, thus preserving the code of
the original method as much as possible.

The first few steps of the algorithm are preparatory ones. Their purpose is to remove some ambiguities which may appear
in later steps of the algorithm. The first preparatory step consists of renaming some of the local variables in the
method body such that no name clashes are generated later on. Then, \code{foreach} loops containing at least one
recursive call are converted to explicit \code{for} loops, because otherwise the control flow graph could not be built.
Single statements which may appear as the branches of \code{if} statements or as the bodies of \code{for}, \code{while}
and \code{do-while} statements are converted to block statements containing the original statements, because some steps
of the algorithm may expand the original statements to multiple ones.

Since a recursive call may be part of a bigger expression and the first statement to be executed after returning from
the recursive call has to be separated from the call itself, all the recursive calls get extracted to their own
statements (if they are not already part of their own statement), by declaring a local variable initialized with the
value of the recursive call and replacing the original call with a reference to this variable. Recursive calls of
methods with \code{void} return type are already separated from other statements.

A new static nested class containing fields corresponding to the local variables of the method is declared inside the
class containing the recursive method. Instances of this class are pushed to and popped from a stack object which
simulates the call stack of the original method. For each recursive call, including the first call of the recursive
method by another method, there is a corresponding \code{push} call to the stack. For each implicit or explicit return
from a recursive call, there is a corresponding \code{pop} call from the stack.

Before further steps of the algorithm can take place, the body of the original method is incorporated inside the
statements which will simulate the call stack, in order to ensure the consistent state of the code between further
transformations. This transformation is necessary in order to be able to make use of the framework of the IDE inside
which this refactoring is implemented, which imposes some restrictions.

Each reference to a local variable (including formal parameters of the method) is then replaced with an access to the
corresponding field of the \code{frame} object at the top of the simulated stack. Local variable declarations having
initializers from the original method body are also replaced by assignments to the corrresponding fields of the
\code{frame} object.

The control flow graph of the method is then generated using a visitor which performs specific tasks for each type of
node in the abstract syntax tree representing the body of the method. The visitor takes special care of the statements
affecting control flow, including \code{if} statements, looping constructs (\code{for}, \code{while}, \code{do-while},
\code{switch} statements) and also \code{break}, \code{continue} and \code{return} statements. Before replacing the
original body of the method with a list of \code{case} labels, one for each basic block in the control flow graph,
the unreachable blocks which may have been previously generated are removed. The trivial blocks containing only jumps to
other blocks are also replaced with jumps from the previous blocks directly to the blocks to which the trivial blocks
jump. The blocks with only one predecessor which are not generated because of a previous recursive call (but get
generated when visiting statements affecting control flow) can also be recursively inlined inside the predecessor
blocks, without modifying the semantics of the code and thus reducing the number of the generated basic blocks to the
bare mininum and improving the readability of the tranformed code.

The last step of the algorithm is replacing the \code{return} statements with \code{pop} calls from the simulated stack.

\section{Project Description}

The refactoring is implemented as a plugin for the Intellij IDEA IDE (Integrated Development Environment).
\abbrev{IDE}{Integrated Development Environment} There is an open source community edition of the IDE
available\footnote{\url{https://github.com/JetBrains/intellij-community}}. Intellij IDEA features a framework for
static analysis of Java code with many existing code
inspections\footnote{\url{https://www.jetbrains.com/help/idea/code-inspection.html}} which detect compiler and potential
runtime errors, but also potential code inefficiencies. The plugin introduces a new code inspection in the
\textit{Performance Issues} group of inspections for the Java programming language. The default severity of the inspection is
\textit{Warning}. The \textit{Remove Recursion} inspection can be seen in \labelindexref{Figure}{img:inspection-settings}.

The Intellij Platform SDK (Software Development Kit)\abbrev{SDK}{Software Development Kit}\footnote{\url{http://www.jetbrains.org/intellij/sdk/docs/}}
presents the API (Application Programming Interface)\abbrev{API}{Application Programming Interface} for manipulating
the source code in a file. A PSI (Program Structure Interface)\abbrev{PSI}{Program Structure Interface}
File\footnote{\url{http://www.jetbrains.org/intellij/sdk/docs/basics/architectural_overview/psi_files.html}} is the root
of a structure representing the contents of a file as a hierarchy of PSI elements. A PSI
element\footnote{\url{http://www.jetbrains.org/intellij/sdk/docs/basics/architectural_overview/psi_elements.html}} can
have child PSI elements. The \code{PsiElement} class is the common base class for all PSI elements. There is a
\code{PsiElement} subclass for each element in the Java programming language.

In Intellij IDEA, new Java code inspections are created by extending the abstract class\\
\code{com.siyeh.ig.BaseInspection}\footnote{\url{https://github.com/JetBrains/intellij-community/blob/master/plugins/InspectionGadgets/InspectionGadgetsAnalysis/src/com/siyeh/ig/BaseInspection.java}}.
Each code inspection needs to override the abstract method \code{buildVisitor()} in \code{com.siyeh.ig.BaseInspection}
to provide a custom instance of \code{com.siyeh.ig.BaseInspectionVisitor}\footnote{\url{https://github.com/JetBrains/intellij-community/blob/master/plugins/InspectionGadgets/InspectionGadgetsAnalysis/src/com/siyeh/ig/BaseInspectionVisitor.java}},
which visits each element of a PSI File and detects problems specific to that inspection. Depending on the inspection
severity, the problem is highlighted in the editor accordingly. In the case of the recursion removal inspection,
the visitor verifies each method call expression to see if it resolves to the containing method and if the qualifier
is either absent or \code{this}. If this is the case, then the visitor registers an error on the method call expression
and it appears highlighted in the editor as in \labelindexref{Figure}{img:recursion-highlight-warning}.

\fig[width=4in]{src/img/inspection-settings.png}{img:inspection-settings}{Remove Recursion inspection}
\fig[width=4in]{src/img/recursion-highlight-warning.png}{img:recursion-highlight-warning}{Recursion Highlighted as a Warning}