\chapter{My Introduction}

The Java programming language compiler \code{javac} does not perform any kind of recursion removal. Recursive methods
are known to potentially exhaust the call stack space. The purpose of this thesis is to create a refactoring which
automatically transforms a recursive method into an iterative one, by simulating the call stack with a stack in the user
program. The refactoring alters the initial recursive method as little as possible while still preserving the semantics
of the code.

\section{Project Description}

The refactoring will be implemented as a plugin for the Intellij IDEA IDE (Integrated Development Environment).
\abbrev{IDE}{Integrated Development Environment} There is an open source community edition of the IDE
available\footnote{\url{https://github.com/JetBrains/intellij-community}}. Intellij IDEA features a framework for
static analysis of Java code with many existing code
inspections\footnote{\url{https://www.jetbrains.com/help/idea/code-inspection.html}} which detect compiler and potential
runtime errors, but also potential code inefficiencies. The plugin introduces a new code inspection in the
\textit{Performance Issues} group of inspections for the Java programming language. The default severity of the inspection is
\textit{Warning}. The \textit{Remove Recursion} inspection can be seen in \labelindexref{Figure}{img:inspection-settings}.

The Intellij Platform SDK (Software Development Kit)\abbrev{SDK}{Software Development Kit}\footnote{\url{http://www.jetbrains.org/intellij/sdk/docs/}}
presents the API (Application Programming Interface)\abbrev{API}{Application Programming Interface} for manipulating
the source code in a file. A PSI (Program Structure Interface)\abbrev{PSI}{Program Structure Interface}
File\footnote{\url{http://www.jetbrains.org/intellij/sdk/docs/basics/architectural_overview/psi_files.html}} is the root
of a structure representing the contents of a file as a hierarchy of PSI elements. A PSI
element\footnote{\url{http://www.jetbrains.org/intellij/sdk/docs/basics/architectural_overview/psi_elements.html}} can
have child PSI elements. The PsiElement class is the common base class for all PSI elements. There is a PsiElement
subclass for each element in the Java programming language.

In Intellij IDEA, new Java code inspections are created by extending the abstract class\\
\code{com.siyeh.ig.BaseInspection}\footnote{\url{https://github.com/JetBrains/intellij-community/blob/master/plugins/InspectionGadgets/InspectionGadgetsAnalysis/src/com/siyeh/ig/BaseInspection.java}}.
Each code inspection needs to override the abstract method \code{buildVisitor()} in \code{com.siyeh.ig.BaseInspection}
to provide a custom instance of \code{com.siyeh.ig.BaseInspectionVisitor}\footnote{\url{https://github.com/JetBrains/intellij-community/blob/master/plugins/InspectionGadgets/InspectionGadgetsAnalysis/src/com/siyeh/ig/BaseInspectionVisitor.java}},
which visits each element of a PSI File and detects problems specific to that inspection. Depending on the inspection
severity, the problem is highlighted in the editor accordingly. In the case of the recursion removal inspection,
the visitor verifies each method call expression to see if it resolves to the containing method and if the qualifier
is either absent or this. If this is the case, then the visitor registers an error on the method call expression and it
appears highlighted in the editor as in \labelindexref{Figure}{img:recursion-highlight-warning}.

\fig[width=4in]{src/img/inspection-settings.png}{img:inspection-settings}{Remove Recursion inspection}
\fig[width=4in]{src/img/recursion-highlight-warning.png}{img:recursion-highlight-warning}{Recursion Highlighted as a Warning}