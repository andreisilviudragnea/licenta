\chapter{Prerequisites}

\section{Defining a recursive method in Java}

The JLS (Java Language Specification)\nocite{jls}\abbrev{JLS}{Java Language Specification}\footnote{\url{https://docs.oracle.com/javase/specs/jls/se8/html/jls-15.html#jls-15.12}}
states the following:
\begin{quote}
    Resolving a method name at compile time is more complicated than resolving a field name because of the possibility
    of method overloading. Invoking a method at run time is also more complicated than accessing a field because of the
    possibility of instance method overriding.
\end{quote}
Given the complexity of the method overloading and overriding mechanisms, the definition of a recursive method in this
paper is restricted in such a way in order to avoid these mechanisms.

A \textit{recursive method} is defined as a method whose body contains one or more method call expressions for which the
static method in \labelindexref{Figure}{img:is-recursive} returns \code{true}.

\fig[width=5in]{src/img/isRecursive-white.png}{img:is-recursive}{Test for a recursive method call expression}

For performance reasons, Intellij IDEA code base contains many fail-fast methods. This method is also a fail-fast one.
The first check verifies if the name of the method in the method call expression matches the name of the expected
method. Only after passing this simple check does the test verify that the method to which the call expression resolves
is indeed the expected method. Resolving the method is an expensive operation compared with a string equality test.

The \code{PsiMethod} instance returned by the call to \code{resolveMethod()} represents the
\textit{compile-time declaration}\footnote{\url{https://docs.oracle.com/javase/specs/jls/se8/html/jls-15.html#jls-15.12.3}}
for the method invocation \code{expression}. In order to guarantee that the method which gets called at runtime is the
same as the method corresponding to the compile-time declaration, further checks are needed.
JLS specifies that\footnote{\url{https://docs.oracle.com/javase/specs/jls/se8/html/jls-15.html#jls-15.12.3}}:
\begin{quote}
    If the compile-time declaration has the static modifier, then the invocation mode is static.
    Otherwise, if the compile-time declaration has the private modifier, then the invocation mode is nonvirtual.
\end{quote}
It also specifies that\footnote{\url{https://docs.oracle.com/javase/specs/jls/se8/html/jls-15.html#jls-15.12.4.4}}:
\begin{quote}
    The strategy for method lookup depends on the invocation mode.
    If the invocation mode is static, no target reference is needed and overriding is not allowed. Method m of class T is the one to be invoked.
    If the invocation mode is nonvirtual, overriding is not allowed. Method m of class T is the one to be invoked.
\end{quote}

What this means is that in case of \code{static} and \code{private} methods, the method invoked at runtime is the method
corresponding to the compile-time declaration. The last check of the test method corresponds to the case when the method
is an non-\code{private} instance method (not \code{static}). In order to avoid the overriding mechanism, the only
methods which this refactoring will consider are those which contain unqualified (or qualified with \code{this}) method
call expressions. This means that the \textit{target reference} of the method remains unmodified.

\section{Fields of the frame class}

When calling another method, the caller creates a new stack frame in which the callee will
execute\footnote{\url{http://docs.oracle.com/javase/specs/jvms/se7/html/jvms-2.html#jvms-2.6}}. This is done
in order to restore the state of the local variables in the caller when the callee returns. The JVM also stores all
the parameters of a method as local
variables\footnote{\url{http://docs.oracle.com/javase/specs/jvms/se7/html/jvms-2.html#jvms-2.6.1}}. The length of the
local variables array is determined at compile-time and the storage for the variables lives until the method returns.

Given the fact that the output of the refactoring is still Java code and not bytecode, the restrictions on what
variables need to be saved in the frame object are not so tight. However, the scope of the variables needs to be taken
into account with respect to the recursive calls in order to decide which variables need to be saved.

In particular, the only variables (formal parameters and local variables) that need to be saved in the current frame are
those whose scope contains at least one recursive call. Otherwise, the variables are declared, initialized and their
value is used only before or after the recursive call, so they do not need to be saved.

JLS specifies that\footnote{\url{https://docs.oracle.com/javase/specs/jls/se8/html/jls-6.html#jls-6.3}}:
\begin{quote}
    The scope of a formal parameter of a method, constructor, or lambda expression is the entire body of the method,
    constructor, or lambda expression.

    The scope of a local variable declared in the \textit{FormalParameter} part of an enhanced \code{for} statement
    is the contained \textit{Statement}.
\end{quote}

Because the formal parameters of a recursive method are always in scope in the method body, they always get saved in the
current frame object. However, if the formal parameter is declared in the header of an enhanced \code{for} statement, it
will have a corresponding field in the frame object only if the body of the statement contains at least one recursive
call.

JLS also specifies that\footnote{\url{https://docs.oracle.com/javase/specs/jls/se8/html/jls-6.html#jls-6.3}}:
\begin{quote}
    The scope of a local variable declaration in a block is the rest of the block in which the declaration
    appears, starting with its own initializer and including any further declarators to the right in the local variable
    declaration statement.

    The scope of a local variable declared in the \textit{ForInit} part of a basic \code{for} statement includes all of
    the following:
    \begin{itemize}
        \item Its own initializer
        \item Any further declarators to the right in the \textit{ForInit} part of the \code{for} statement
        \item The \textit{Expression} and \textit{ForUpdate} parts of the \code{for} statement
        \item The contained \textit{Statement}
    \end{itemize}
\end{quote}

A local variable declared in a block will have a corresponding field in the frame object only if its scope contains at
least a recursive call. Intuitively speaking, this is the case when the variable is declared in the initialization part
of a \code{for} statement containing a recursive call, usually in its body, or when the variable is declared in a block
and after the declaration there is a recursive call in one of the remaining statements of the block.